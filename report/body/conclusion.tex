\section{Conclusion}
  The practical element of this paper successfully demonstrated serverless architecture using AWS Lambda. In this case, a serverless architecture was implemented to create a MEAN style web application without the use of any servers. The fronted was provided entirely by S3, serving HTML web pages and client side scripts as a static website. The backend was provided by a Lambda function triggered by API endpoints configured in AWS' API Gateway. Finally, user authentication was provided by AWS Cognito. Therefore, the entire web application was hosted in the cloud without deploying, configuring or maintaining servers. This also has the benefit or reduced costs.
  
  A notable drawback of the system created in this paper is vendor lock-in. This system was deployed as a number of AWS services all configured together to create a single architecture. Migrating a system such as this to another vendor such as Google Cloud Platform or Microsoft Azure could prove to be quite a monumental task. Knowledge and experience would be needed in each of the corresponding services on the new platform before the system can be migrated. 
  
  It is possible that designing this system as a traditional MEAN stack application which can be run on a server with the necessary software installed (e.g. NodeJS) could make it far easier to port to another provider. In order to port such a system only knowledge of the new platforms compute service would be necessary. Once a server was create. the necessary software could be installed and the application started. This approach however, introduces the overhead of server maintainence.
  
  It must also be noted however that vendor lock-in could also be avoided while still employing a serverless architecture. With the correct configurations and permission, the various services needs could be provided by different vendors. For example, the static website server from S3 could be configured to target and API created using Cloud Endpoints, Google Cloud Platform's API service.